\documentclass[a4paper,12pt]{article}
\usepackage{polski}
\usepackage[utf8]{inputenc}
\usepackage[left = 3cm, right = 3cm, top = 2cm, bottom = 2cm]{geometry}
\usepackage{enumerate}
\usepackage{amssymb}		% pakiet do symboli
\usepackage{mathtools}		% pakiet do matmy (rozszerza amsmath)
\usepackage{enumitem}		% punktowanie (a), (b), ...
\usepackage{nopageno}		% brak numerow stron
\usepackage{graphicx}		% wstawianie obrazkow
\usepackage{float}			% wstawianie obrazkow w dowolnym miejscu
\usepackage{caption}
\usepackage{esdiff}         % pochodne \diff{}{}
\usepackage{listings}
\usepackage{xcolor}
\usepackage{adjustbox}

% nowe komendy dla wygodniejszego pisania :)

\newcommand{\floor}[1]{\left\lfloor #1 \right\rfloor}	% podłoga
\newcommand{\ceil}[1]{\left\lceil #1 \right\rceil}		% sufit
\newcommand{\fractional}[1]{\left\{ #1 \right\}}		% część ułamkowa {x}
\newcommand{\abs}[1]{\left| #1 \right|}					% wartosc bezwzgledna / moc
\newcommand{\set}[1]{\left \{ #1 \right \}}				% zbiór elementów {a,b,c}
\newcommand{\pair}[1]{\left( #1 \right)}				% para elementów (a,b)
\newcommand{\Mod}[1]{\ \mathrm{mod\ #1}}				% lekko zmodyfikowane modulo
\newcommand{\comp}[1]{\overline{ #1 }} 					% dopełnienie zbioru 
\newcommand{\annihilator}{\mathbf{E}}					% operator E
\newcommand{\seqAnnihilator}[1]{\annihilator \left\langle #1 \right\rangle} % E(a_n)
\newcommand{\sequence}[1]{\left\langle #1 \right\rangle} % <a_n>
\DeclareMathOperator{\lcm}{lcm}							% obsługa lcm w mathmode

% styl do kodu
\lstdefinestyle{code}{%
basicstyle=\ttfamily\small,
commentstyle=\color{green!60!black},
keywordstyle=\color{magenta},
stringstyle=\color{blue!50!red},
showstringspaces=false,
numbers=left,
numberstyle=\footnotesize\color{gray},
numbersep=10pt,
tabsize=4,
rulecolor=\color{red},
breaklines=true
}

\newcommand{\code}[1]{\lstinline[style=code]{#1}} % kod inline

\begin{document}
\noindent \textbf{Matematyka dyskretna L, Lista 8 - Tomasz Woszczyński}\newline

\noindent \newline \textbf{Zadanie 1} \newline
Niech $A(x)$ będzie funkcją tworzącą ciągu $a_n$. Podaj postać funkcji tworzącej
dla ciągu $s_n = a_0 + a_1 + a_2 + \ldots + a_n$. \textit{Wskazówka}: Trzeba 
użyć funkcji tworzącej $\frac{1}{1-x}$. \\

\noindent Skoro $s_n = a_0 + a_1 + a_2 + \ldots + a_n = \sum\limits_{k=0}^n a_k$, 
to w łatwy sposób możemy podstawić $s_n$ w postaci sumy do wzoru na funkcję tworzącą, 
a następnie całe wyrażenie przekształcić do takiej postaci:
\begin{align*}
    A(x) &= \sum\limits_{n=0}^{\infty} s_n x^n = \sum\limits_{n = 0}^{\infty} 
    \left(\sum\limits_{k=0}^n a_k \right) x^n = \sum\limits_{n=0}^{\infty} 
    \left(\sum\limits_{k=0}^{n} \left( a_k x^k \right) \cdot x^{n-k} \right)  \\
    &= \left(\sum\limits_{n=0}^{\infty} a_n x^n \right) \cdot 
    \left(\sum\limits_{n=0}^{\infty} 1 \cdot x^n \right) = A(x) 
    \cdot \frac{1}{1 - x} = \frac{A(x)}{1 - x}
\end{align*}

\noindent \newline \textbf{Zadanie 2} \newline
Wyznacz funkcje tworzące poniższych ciągów. Wszędzie przyda się funkcja tworząca
$\frac{1}{1-x}$, a w ostatnim podpunkcie będzie to odpowiednia potęga tej funkcji:
\begin{enumerate}
    \item $a_n = n^2$
    \item $a_n = n^3$
    \item $\binom{n+k}{k}$
\end{enumerate}

\noindent \textbf{Przykład 1:} Chcemy uzyskać funkcję tworzącą ciągu 
$\sequence{n^2} = \sequence{0, 1, 4, 9, 16, \ldots}$. Niech $A$ przedstawia
funkcję tworzącą $0 + 1x + 4x^2 + 9x^3 + \ldots$, weźmy wtedy funkcję $-xA$,
którą przedstawia następujący szereg: $-x^2 - 4x^3 - 9x^4 - \ldots$. Dodajmy je:
\[
\begin{array}{ccccccccc}
    A     &=  &0  &+ 1x &+ 4x^2 &+ 9x^3 &+ 16x^4 &+ \ldots & \\
    -xA   &=  &   &     &- 1x^2 &- 4x^3 &- 9x^4  &- \ldots & \\
    \hline
    A-xA  &=  &   &+ 1x &+ 3x^2 &+ 5x^3 &+ 7x^4  &+ \ldots &= B \\
\end{array}
\]
Wiemy, że $A-xA = A(1-x) = B$, a więc $A = \frac{B}{1-x}$. Powtórzmy poprzedni 
krok jeszcze raz, tym razem aplikując zmiany do $B$ i uzyskując $-xB$:
\[
\begin{array}{ccccccccc}
    B     &=  &   & 1x &+ 3x^2 &+ 5x^3 &+ 7x^4  &+ \ldots & \\
    -xB   &=  &   &    &- 1x^2 &- 3x^3 &- 5x^4  &- \ldots & \\
    \hline
    B-xB  &=  &   & 1x &+ 2x^2 &+ 2x^3 &+ 2x^4  &+ \ldots &= B \\
\end{array}
\]
Otrzymaliśmy więc funkcję $B(1-x) = x + 2x^2 + 2x^3 + 2x^4 + \ldots$. Znajdźmy dla 
takiego ciągu funkcję tworzącą, a następnie dla ciągu $B$:
\[ 
    B(1-x) = x + \frac{2x^2}{1 - x} = \frac{2x^2 + x - x^2}{1 - x} = \frac{x(1+x)}{1-x} 
    \Longrightarrow B = \frac{x(1+x)}{(1-x)^2}
\]
Wynik $B$ możemy podstawić do wzoru na $A$, stąd więc mamy 
$A = \frac{B}{1 - x} = \frac{x(1+x)}{(1-x)^3}$.

\newpage
\noindent \textbf{Przykład 2:} W tym przykładzie chcemy uzyskać funkcję tworzącą
ciągu $\sequence{n^3} = \sequence{0, 1, 8, 27, \ldots}$. Podobnie jak w pierwszym 
przykładzie, interesować nas będą pochodne (tam inaczej rozpisane). Tym razem 
potrzebna jest nam pochodna ciągu $\sequence{n^2}$, a następnie przesunięcie 
całego ciągu w prawo.
\[
    x \cdot A'(x) =  x \cdot \frac{x^2 + 4x + 1}{(x-1)^4} = 
    x \cdot (1 + 2^3x^2 + 3^3x^3 + 4^3x^4 + \ldots) = \sequence{0, 1^3, 2^3, 3^3, \ldots} 
\]

\noindent \textbf{Przykład 3:} Teraz szukamy funkcję tworzącą ciągu $\sequence{\binom{n+k}{k}}$.
Kolejnymi wyrazami tego ciągu są $\sequence{1, \binom{k+1}{k}, \binom{k+2}{k}, \binom{k+3}{k},
\ldots}$. Chcemy obliczyć funkcję tworzącą 
\[
    Z(x) = \sum\limits_{k=0}^{\infty} \binom{n+k}{k} x^k
\]
Wiemy, że szereg kolejnych dwumianów Newtona można zwinąć do zwięzłej funkcji:
\[
    \sum\limits_{k=0}^{\infty} \binom{n}{k} x^k = (1+x)^n
\]
Skorzystam również z poniższej własności symbolu Newtona:
\[
    \binom{n}{k} = (-1)^k \binom{-n + k - 1}{k}
    \Longrightarrow \binom{n+k}{k} = (-1)^k \binom{-n - 1}{k}
\]
Łącząc powyższe informacje, przekształćmy wyjściową funkcję tworzącą $Z(x)$, aby
otrzymać jej zwięzłą formę. Korzystamy najpierw z własności symbolu Newtona, później
mnożymy $(-1)^k \cdot x^k = (-x)^k$. Na końcu zwijamy wszystko do jednej sumy,
używając w tym celu uogólnionego symbola dwumianowego. 
\begin{align*}
    Z(x)    &= \sum\limits_{k=0}^{\infty} \binom{n+k}{k} x^k
             = \sum\limits_{k=0}^{\infty} (-1)^k \binom{-n - 1}{k} x^k
             = \sum\limits_{k=0}^{\infty} \binom{-(n+1)}{k} (-x)^k = \\
            &= \left( 1 + (-x) \right)^{-(n+1)} 
             = \frac{1}{(1 - x)^{n+1}}, \text{ co kończy dowód.}
\end{align*}

% \noindent \newline \textbf{Zadanie 3 TODO} \newline
% Oblicz funkcje tworzące ciągów:
% \begin{enumerate}
%     \item $a_n = n$ dla $n$ parzystych oraz $a_n = \frac{1}{n}$ dla nieparzystych $n$,
%     \item $H_n = 1 + \frac{1}{2} + \ldots + \frac{1}{n}$, $H_0 = 0$.
% \end{enumerate}

% \noindent \textbf{Przykład 1:} Musimy obliczyć funkcje tworzącą ciągu:
% \[
%     a_n = 
%     \begin{cases}
%         n           &\text{ dla } n \text{ parzystych} \\
%         \frac{1}{n} &\text{ dla } n \text{ nieparzystych}
%     \end{cases}  
% \]
% Kolejnymi wyrazami tego ciągu są $\sequence{0, \frac{1}{1}, 2, \frac{1}{3}, 4, \frac{1}{5},
% \ldots}$. Do obliczenia 

\newpage
\noindent \textbf{Zadanie 6} \newline
Niech $Q_k$ oznacza graf $k$-wymiarowej kostki, tzn. zbiór wierzchołków tego grafu
tworzą wszystkie $k$-elementowe ciągi zer i jedynek i dwa wierzchołki są sąsiednie
tylko wtedy, gdy odpowiadające im ciągi różnią się dokładnie jedną współrzędną.
Oblicz, ile wierzchołków i krawędzi ma graf $Q_k$. \\

\noindent Obliczenie liczby wierzchołków grafu $Q_k$ jest łatwe - są to wszystkie możliwe
permutacje $k$-bitowych słów na słowniku $\Sigma = \{ 0, 1\}$, więc 
\[ 
    \abs{V} = 2^k
\]

\noindent Aby obliczyć liczbę krawędzi grafu $Q_k$, musimy wziąć pod uwagę to, że
ten graf jest nieskierowany, a więc $(u, v)$ i $(v, u)$ to jedna krawędź (dlatego
dzielimy liczbę wszystkich krawędzi przez $2$). Z każdego $k$-bitowego wierzchołka
można wyprowadzić $k$ krawędzi, np. dla $4$-bitowego wierzchołka $0001$ będą to 
$1001$, $0101$, $0011$ i $0000$. Mamy więc 
\[
    \abs{E} = \frac{k \cdot 2^k}{2} = k \cdot 2^{k-1}
\]

\noindent \newline \textbf{Zadanie 7} \newline
Problem izomorfizmu dwóch grafów jest trudny. Załózmy natomiast, że w komputerze
są dane dwa grafy $G$ i $H$, określone na tym samym zbiorze wierzchołków
$V(G) = V(H) = \{ 1, 2, 3, \ldots, n \}$. Niech $m$ oznacza liczbę krawędzi grafu
$G$. Podaj algorytm sprawdzający w czasie $O(m+n)$, czy te grafy są identyczne.

\begin{lstlisting}[style=code, language=python]
def is_identical(G, H):
    visited = [False, ..., False] # length n

    for vertex in vertices:
        nodes_diff = 0

        for neighbour in G[vertex]:
            visited[vertex] = True
            nodes_diff += 1

        for neighbour in H[vertex]:
            if visited[vertex] == False:
                return False
            
            visited[vertex] = False
            nodes_diff -= 1

        if nodes_diff != 0:
            return False

    return True
\end{lstlisting}

\noindent Algorytm polega na tym, że dla każdego wierzchołka ze zbioru $V$ dla
grafu $G$ sprawdzamy, czy istnieje taki sam sąsiad w grafie $H$. Ponadto w tablicy
\code{visited} zapisujemy informacje o tym, czy wierzchołek był już odwiedzony.
Zadaniem \code{nodes_diff} jest porównanie stopni wierzchołka \code{vertex} 
w każdym grafie - jeżeli liczba ta jest różna od $0$, to grafy nie mogą być 
identyczne. W przeciwnym wypadku, po sprawdzeniu wszystkich wierzchołków, grafy
na pewno są identyczne.

\newpage
\noindent \textbf{Zadanie 8} \newline
Rozważ reprezentacje grafu $G$: macierzową (za pomocą macierzy sąsiedztwa), 
listową. Dla każdej z tych reprezentacji określ złożoność wykonania na grafie
$G$ nastepujących operacji:
\begin{enumerate}
    \item oblicz stopień ustalonego wierzchołka,
    \item przeglądnij wszystkie krawędzie grafu,
    \item sprawdź, czy krawędź $(u, v)$ należy do grafu $G$,
    \item usuń z grafu $G$ krawędź $(u, v)$,
    \item wstaw do grafu $G$ krawędź $(u, v)$.
\end{enumerate}

\begin{table}[h]
\centering
\begin{adjustbox}{width=1\textwidth}
\tiny
\begin{tabular}{|c|c|c|}
    \hline
    Operacja                   & Repr. macierzowa & Repr. listowa           \\ \hline
    obliczenie $\deg(v)$       &  $O(n)$          & $O(\deg(v))$            \\ \hline
    przejrzenie krawędzi $E$   &  $O(n^2)$        & $O(n+m)$                \\ \hline
    czy krawędź należy do $G$  &  $O(1)$          & $O(\deg(u))$            \\ \hline
    usuń krawędź z grafu $G$   &  $O(1)$          & $O(\deg(u))$            \\ \hline
    dodaj krawędź do grafu $G$ &  $O(1)$          & $O(\deg(u) + \deg(v))$  \\ \hline
\end{tabular}
\end{adjustbox}
\end{table}

\noindent \newline 1. Aby obliczyć stopień wierzchołka $v$ należy przejrzeć cały 
wiersz w reprezentacji macierzowej i dodać elementy o wartości $1$ (istniejący sąsiad),
a więc zajmie to $O(n)$ czasu, a dla listy będzie to $O(\deg(v))$, gdyż w reprezentacji
listowej przechowywani są sąsiedzi wierzchołka $v$, których jest tyle, ile wynosi $\deg(v)$.

\noindent \newline 2. Przejrzenie wszystkich krawędzi wymaga przejrzenia całej macierzy, 
więc $O(n^2)$, a w reprezentacji listowej listy sąsiadów każdego wierzchołka, czyli $O(n+m)$.

\noindent \newline 3. Przynależność krawędzi do grafu zajmuje czas $O(1)$ w reprezentacji 
macierzowej, wystarczy sprawdzić, czy np. \code{E[i][j] == 1}. W przypadku reprezentacji
listowej należy sprawdzić całą listę sąsiadów, więc zajmuje to $O(\deg(u))$.

\noindent \newline 4. Usunięcie krawędzi z $G$ zajmuje czas $O(1)$ w przypadku macierzowej
reprezentacji, podobnie jak sprawdzenie przynależności do grafu, zmieniamy tylko wartość
odpowiedniej komórki z $1$ na $0$. W przypadku reprezentacji listowej będzie to $O(\deg(u))$. 

\noindent \newline 5. Dodanie krawędzi do grafu w reprezentacji macierzowej działa 
tak samo jak usuwanie i sprawdzenie przynależności do grafu $G$, więc zajmuje $O(1)$.
W przypadku reprezentacji listowej jest to $O(\deg(u))$ dla grafu skierowanego lub
$O(\deg(u) + \deg(v))$ dla grafu nieskierowanego (dodajemy krawędź do obu list sąsiadów). 

\end{document}
\documentclass[a4paper,12pt]{article}
\usepackage{polski}
\usepackage[utf8]{inputenc}
\usepackage[left = 3cm, right = 3cm, top = 2cm, bottom = 2cm]{geometry}
\usepackage{enumerate}
\usepackage{amssymb}		% pakiet do symboli
\usepackage{amsmath}		% pakiet do matmy
\usepackage{enumitem}		% punktowanie (a), (b), ...
\usepackage{nopageno}		% brak numerow stron
\usepackage{graphicx}		% wstawianie obrazkow
\usepackage{float}			% wstawianie obrazkow w dowolnym miejscu
\usepackage{titling}

% nowe komendy dla wygodniejszego pisania :)
\newcommand{\subtitle}[1]{ \posttitle{ \par\end{center} \begin{center}\large#1\end{center} \vskip0.5em}}
\newcommand{\floor}[1]{\left\lfloor #1 \right\rfloor}
\newcommand{\ceil}[1]{\left\lceil #1 \right\rceil}

\begin{document}
\noindent \textbf{Lista 2, zadanie 6 - Tomasz Woszczyński}\newline

\noindent \newline \textbf{Treść:} Ułóż algorytm, który dla danego spójnego grafu $G$ oraz krawędzi $e$ sprawdza w czasie $O(n+m)$, czy krawędź $e$ należy do jakiegoś minimalnego drzewa spinającego grafu $G$. Możesz założyć, że wszystkie wagi krawędzi są różne. \newline

\noindent \textbf{Rozwiązanie:} Skorzystamy z cycle property dla MST mówiącego o tym, że dla każdego cyklu $C$, jeśli waga krawędzi $e$ należącej do $C$ jest większa od pozostałych wag w $C$, to ta krawędź \textbf{nie może} być w MST. Załóżmy więc, że $e$ nie jest maksymalna na żadnym cyklu w grafie $G$ i nie należy do MST. Rozpatrzmy więc przypadki:
\begin{enumerate}
\item krawędź $e$ nie leży na żadnym cyklu: stąd oczywiste jest, że $e$ należy do MST.
\item krawędź $e$ leży na jakimś cyklu $C$: weźmy MST i dołóżmy do niego krawędź $e$ - po tym kroku powstaje nam cykl, jako że $e$ nie należało do MST. W tym drzewie rozpinającym musi istnieć jakaś krawędź maksymalna $\hat{e}$. Po usunięciu jej otrzymamy MST o mniejszej wadze, a więc dochodzimy do sprzeczności (MST to drzewo rozpinające o najmniejszej wadze).
\end{enumerate}

\noindent \textbf{Algorytm:} Wiemy, że krawędź $e$ łączy wierzchołki $v, w$. Zapamiętajmy wagę tej krawędzi i ją usuńmy. Z wierzchołka $v$ puszczamy DFS, ale sprawdzamy jedynie wierzchołki połączone krawędziami o wagach nie większych od $e$. Jeśli dotrzemy do $w$, wypisujemy NIE, w przeciwnym wypadku wypisujemy TAK. \newline

\noindent \textbf{Wyjaśnienie:} Jeśli nie dotarliśmy do $w$, to albo $e$ jest mostem, albo $e$ nie była maksymalna na żadnym cyklu, bo gdyby była, to przeszlibyśmy po każdej krawędzi tych cykli pomijając krawędź $e$. Złożoność czasową $O(n+m)$ osiągamy w bardzo łatwy sposób, gdyż wiemy, że DFS działa właśnie w takim czasie.



\end{document}
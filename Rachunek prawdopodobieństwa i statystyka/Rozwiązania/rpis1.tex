\documentclass[a4paper,12pt]{article}
\usepackage{polski}
\usepackage[utf8]{inputenc}
\usepackage[left = 3cm, right = 3cm, top = 2cm, bottom = 2cm]{geometry}
\usepackage{enumerate}
\usepackage{amssymb}		% pakiet do symboli
\usepackage{mathtools}		% pakiet do matmy (rozszerza amsmath)
\usepackage{enumitem}		% punktowanie (a), (b), ...
\usepackage{nopageno}		% brak numerow stron
\usepackage{graphicx}		% wstawianie obrazkow
\usepackage{float}			% wstawianie obrazkow w dowolnym miejscu
\usepackage{caption}
\usepackage{esdiff}         % pochodne \diff{}{}
\usepackage{listings}
\usepackage{xcolor}
\usepackage{adjustbox}
\usepackage{tkz-graph}
%\usepackage[none]{hyphenat} % usunięcie łamania wyrazów na końcu linii

% nowe komendy dla wygodniejszego pisania :)

\newcommand{\floor}[1]{\left\lfloor #1 \right\rfloor}	% podłoga
\newcommand{\ceil}[1]{\left\lceil #1 \right\rceil}		% sufit
\newcommand{\fractional}[1]{\left\{ #1 \right\}}		% część ułamkowa {x}
\newcommand{\abs}[1]{\left| #1 \right|}					% wartosc bezwzgledna / moc
\newcommand{\set}[1]{\left \{ #1 \right \}}				% zbiór elementów {a,b,c}
\newcommand{\pair}[1]{\left( #1 \right)}				% para elementów (a,b)
\newcommand{\Mod}[1]{\ \mathrm{mod\ #1}}				% lekko zmodyfikowane modulo
\newcommand{\comp}[1]{\overline{ #1 }} 					% dopełnienie zbioru 
\newcommand{\annihilator}{\mathbf{E}}					% operator E
\newcommand{\seqAnnihilator}[1]{\annihilator \left\langle #1 \right\rangle} % E(a_n)
\newcommand{\sequence}[1]{\left\langle #1 \right\rangle} % <a_n>
\DeclareMathOperator{\lcm}{lcm}							% obsługa lcm w mathmode

% styl do kodu
\lstdefinestyle{code}{%
basicstyle=\ttfamily\small,
commentstyle=\color{green!60!black},
keywordstyle=\color{magenta},
stringstyle=\color{blue!50!red},
showstringspaces=false,
numbers=left,
numberstyle=\footnotesize\color{gray},
numbersep=10pt,
tabsize=4,
rulecolor=\color{red},
breaklines=true
}

\newcommand{\code}[1]{\lstinline[style=code]{#1}} % kod inline

\begin{document}
\noindent \textbf{RPiS, Lista 1 - Tomasz Woszczyński}\newline

\noindent \newline \textbf{Zadanie 1} \newline
Sprawdzić, że:
\begin{enumerate}[label=(\alph*)]
    \item $\sum\limits_{k=0}^{n} \binom{n}{k} p^k (1-p)^{n-k} = 1$,
    \item $\sum\limits_{k=0}^{n} k \binom{n}{k} p^k (1-p)^{n-k} = np$.
\end{enumerate}

\noindent \textbf{Podpunkt (a):} skorzystajmy z dwumianu Newtona, a więc ze wzoru
\[
    (x+y)^n = \sum\limits_{k=0}^{n} \binom{n}{k} x^{n-k}y^{k}    
\]
i weźmy $x = 1-p$ oraz $y = p$. Wtedy, po podstawieniu mamy:
\[
    \sum\limits_{k=0}^{n} \binom{n}{k} p^k (1-p)^{n-k} 
    = \left( \left( 1 - p \right) + p \right)^n = 1^n = 1,
    \text{ co należało dowieść.} 
\]

\noindent \textbf{Podpunkt (b):} skorzystajmy z własności 
$k \binom{n}{k} = n \binom{n-1}{k-1}$. Przekształćmy wzór:
\begin{align*}
    \sum\limits_{k=0}^{n} \underbrace{k \binom{n}{k} p^k (1-p)^{n-k}}
    _{\text{dla } k = 0 \text{ jest to } 0} &=
    \sum\limits_{k=1}^{n} n \binom{n-1}{k-1} p^k (1-p)^{n-k} = \\
    &= \sum\limits_{k=1}^{n} n \binom{n-1}{k-1} p \cdot p^{k-1} (1-p)^{n-k} = \\
    &= np \sum\limits_{k=1}^{n} \binom{n-1}{k-1} p^{k-1} (1-p)^{n-k} = \\
    &= np \cdot \left( \left( 1 - p \right) + p \right)^n = 1^n = np \cdot 1^n = np
\end{align*}
Dowiedliśmy więc, że $\sum\limits_{k=0}^{n} k \binom{n}{k} p^k (1-p)^{n-k} = np$.

\noindent \newline \textbf{Zadanie 2} \newline
Sprawdzić, że:
\begin{enumerate}[label=(\alph*)]
    \item $\sum\limits_{k=0}^{\infty} e^{-\lambda} \cdot \frac{\lambda^k}{k!} = 1$,
    \item $\sum\limits_{k=0}^{\infty} k \cdot e^{-\lambda} \cdot \frac{\lambda^k}{k!} = \lambda$.
\end{enumerate}

\noindent \textbf{Podpunkt (a):}
\begin{align*}
    \sum\limits_{k=0}^{\infty} e^{-\lambda} \cdot \frac{\lambda^k}{k!} &= 
    e^{-\lambda} \cdot \sum\limits_{k=0}^{\infty} \frac{\lambda^k}{k!} = \\ 
    &= e^{-\lambda} \cdot \underbrace{\left( \lambda^0 \cdot \frac{1}{0!} 
    + \lambda^1 \cdot \frac{1}{1!} + \lambda^2 \cdot \frac{1}{2!} 
    + \lambda^3 \cdot \frac{1}{3!} + \ldots \right)}_{\text{szereg Maclaurina na }
    e^\lambda} = \\
    &= e^{-\lambda} \cdot e^\lambda = e^{\lambda - \lambda} = e^0 = 1
\end{align*}
\noindent Szereg Maclaurina na $e^\lambda$ wynika z tego, że $e^\lambda = f(\lambda)
= f'(\lambda) = f''(\lambda) = \ldots = 1 = f(0) = f'(0) = f''(0) = \ldots$, czyli 
powyższe przekształcenie jest dowodem wzoru.\\

\noindent \textbf{Podpunkt (b):}
\begin{align*}
    \sum\limits_{k=0}^{\infty} k \cdot e^{-\lambda} \cdot \frac{\lambda^k}{k!} &=
    e^{-\lambda} \cdot \sum\limits_{k=1}^{\infty} \frac{\lambda^k}{(k-1)!} = \\
    &= \lambda \cdot e^{-\lambda} \cdot \sum\limits_{k=1}^{\infty} 
    \frac{\lambda^{k-1}}{(k-1)!} = \\
    &= \lambda \cdot e^{-\lambda} \cdot e^{\lambda} = \lambda
\end{align*}

\noindent \newline \textbf{Zadanie 3} \newline
\textbf{Funkcją $\Gamma$-Eulera} nazywamy wartość całki:
\[
    \Gamma (p) = \int\limits_{0}^{\infty} t^{p-1}e^{-t} dt, \ p > 0    
\]
Wykazać, że $\Gamma(n) = (n-1)!$ dla $n \in \mathbb{N}$. \\

\noindent Udowodnimy to indukcyjnie po $n$:
\begin{enumerate}
    \item Baza indukcyjna: $n=1$, wtedy:
    \[ 
        \Gamma(1) = \int\limits_{0}^{\infty} t^{1-1} e^{-t} dt = 
        \int\limits_{0}^{\infty} e^{-t} dt =
        -e^{-t} \Big{|}_0^\infty = 0 - (-1) = 1 = 0!, \text{ czyli prawda } \checkmark
    \]
    \item Krok indukcyjny: załóżmy, że dla $n \in \mathbb{N}$ jest $\Gamma(n) = (n-1)!$,
          musimy pokazać, że dla $\Gamma(n+1) = n! = n\Gamma(n)$.
    \begin{align*}
        \Gamma(n+1) &= \int\limits_{0}^{\infty} t^{n} e^{-t} dt =
        \begin{vmatrix}
            u = t^n         &   du = nt^{n-1}   \\
            dv = e^{-t}     &   v = -e^{-t}       
        \end{vmatrix} = \\
        &= \underbrace{- \frac{t^n}{e^t} \Big|_0^\infty}_{0}
        + \underbrace{\int\limits_{0}^{\infty} t^{n-1} n e^{-t} dt}_{
            n \int\limits_{0}^{\infty} t^{n-1} e^{-t} dt
        } 
        = n\Gamma(n) = n \cdot (n-1)! = n!
    \end{align*}
\end{enumerate}

\noindent Udowodniliśmy więc, że dla wszystkich dodatnich liczb naturalnych 
$\Gamma(n) = (n-1)!$.

\vspace*{\fill}
\noindent\rule{\textwidth}{1pt}
\noindent Całkowanie przez części: $\int udv = uv - \int vdu$, bierzemy łatwiejszą
do zróżniczkowania część jako $u$ oraz łatwiejszą do scałkowania część $dv$.


\newpage
\noindent \textbf{Zadanie 4} \newline
Niech $f(x) = \lambda \exp (-\lambda x)$, gdzie $\lambda > 0$. Obliczyć wartość całek:
\begin{enumerate}[label=(\alph*)]
    \item $\int\limits_{0}^{\infty} f(x) dx$,
    \item $\int\limits_{0}^{\infty} x f(x) dx$.
\end{enumerate}

\noindent \textbf{Podpunkt (a):}
\begin{align*}
    \int\limits_{0}^{\infty} f(x) dx &= 
        \int\limits_{0}^{\infty} \lambda \exp (-\lambda x) dx =
        \lambda \int\limits_{0}^{\infty} \exp (-\lambda x) dx =
        \lambda \cdot \frac{-\exp(-\lambda x)}{\lambda} \Big|_0^\infty = \\
    &= -\exp(-\lambda x) \Big|_0^\infty =
        \underbrace{\lim\limits_{b \to \infty} \exp(-\lambda b)}_{0} 
        - \underbrace{(-\exp(0))}_{-1} = 0 - (-1) = 1
\end{align*}

\noindent \textbf{Podpunkt (b):}
\begin{align*}
    \int\limits_{0}^{\infty} x f(x) dx &= 
        \int\limits_{0}^{\infty} x \lambda \exp (-\lambda x) dx =
        \lambda \int\limits_{0}^{\infty} x \exp (-\lambda x) dx = \\
    &= 
    \begin{vmatrix}
        u = x                                   &   du = 1   \\
        v = -\frac{\exp(-\lambda x)}{\lambda}   &   dv = \exp(-\lambda x)
    \end{vmatrix} = \\ 
    &= \lambda \left( 
            \left[ \frac{x \exp(-\lambda x)}{\lambda} \right]_0^\infty 
            - \int\limits_{0}^{\infty} -\frac{\exp(-\lambda x)}{\lambda} dx 
        \right) = \\
    &= \lambda \left(
        \left[ - \frac{x\exp(-\lambda x)}{\lambda} \right]_0^\infty
        + \left[ \frac{\exp(-\lambda x)}{\lambda^2} \right]_0^\infty
    \right) = \\
    &= \left[ \frac{(\lambda x + 1) \exp(-\lambda x)}{\lambda} \right]_0^\infty =
        \frac{1}{\lambda}
\end{align*}

\noindent \textbf{Zadanie 5} \newline
Wykazać, że $D_n = n$, gdzie
\[
    D_n =
    \begin{vmatrix}
        1       & -1    & -1    & \dots     & -1    \\
        1       &  1    &       &           &       \\
        1       &       &  1    &           &       \\
        \vdots  &       &       & \ddots    &       \\
        1       &       &       &           & 1    
    \end{vmatrix}
\]

\noindent Dodajmy do pierwszego wiersza wszystkie kolejne, a więc od drugiego do
$n$-tego. Wtedy otrzymamy następującą macierz:
\[
    D_n =
    \begin{vmatrix}
        n       &  0    &  0    & \dots     & 0     \\
        1       &  1    &       &           &       \\
        1       &       &  1    &           &       \\
        \vdots  &       &       & \ddots    &       \\
        1       &       &       &           & 1    
    \end{vmatrix}
\]

\noindent Otrzymana macierz jest macierzą dolnotrójkątną, a więc obliczenie 
wyznacznika tej macierzy sprowadza się do przemnożenia wszystkich wartości
z przekątnej. Otrzymamy więc $\det(D_n) = n \cdot 1 \cdot \ldots \cdot 1 = n$,
co kończy dowód.

\newpage
\noindent \textbf{Zadanie 6} \newline
Niech $I = \int\limits_{-\infty}^{\infty} \exp \left\{ -\frac{x^2}{2} \right\} dx$.
Mamy $I^2 = \int\limits_{-\infty}^{\infty} \int\limits_{-\infty}^{\infty} \exp 
\left\{ -\frac{x^2 + y^2}{2} \right\} dy dx$. Stosując podstawienie $x = r \cos 
\theta$, $y = r \sin \theta$, wykazać, że $I^2 = 2 \pi$. \\

\noindent \textbf{Rozwiązanie z ćwiczeń:} stosując podstawienie z polecenia,
wiemy, że $\theta \in [0, 2\pi)$ oraz $r \in [0, \infty)$. Dzięki temu możemy
policzyć następujący Jakobian:
\[
    \begin{vmatrix}
        \cos \theta     &   -r \sin \theta \\
        \sin \theta     &   r \cos \theta
    \end{vmatrix} = r   
\]
\noindent Wykorzystamy go do policzenia całki podwójnej $I^2$ następująco:
\begin{align*}
    I^2 &= \int\limits_{0}^{\infty} \int\limits_{0}^{2\pi}
            r \exp \left\{ -\frac{r^2}{2} \right\} d\theta dr =
            \int\limits_{0}^{\infty} r \exp \left\{ -\frac{r^2}{2} \right\} dr
            \cdot \int\limits_{0}^{2\pi} 1 d\theta = \\
        &= \left( -\exp \left\{ \frac{-r^2}{2} \right\} \right)_0^\infty
            \cdot 2\pi = 2\pi, \text{ c.n.d.}
\end{align*}


\noindent \textbf{Zadanie 7} \newline
Symbol $\bar{s}$ oznacza średnią ciągu $s_1, \ldots, s_n$. Udowodnić, że:
\begin{enumerate}[label=(\alph*)]
    \item $\sum\limits_{k=1}^{n} (x_k - \bar{x})^2 = 
           \sum\limits_{k=1}^{n} x_k^2 - n \cdot \bar{x}^2$,
    \item $\sum\limits_{k=1}^{n} (x_k - \bar{x})(y_k - \bar{y}) =
           \sum\limits_{k=1}^{n} x_k y_k - n \bar{x} \bar{y}$.
\end{enumerate}

\noindent \textbf{Podpunkt (a):}
\begin{align*}
    \sum\limits_{k=1}^{n} (x_k - \bar{x})^2 &= 
        \sum\limits_{k=1}^{n} (x_k^2 - 2 x_k \bar{x} + \bar{x}^2) = \\
    &= \sum\limits_{k=1}^{n} x_k^2 - 2 x_1 \bar{x} - 2 x_2 \bar{x}
        - 2 x_3 \bar{x} - \ldots - 2 x_n \bar{x} + n \cdot \bar{x}^2 = \\
    &= \sum\limits_{k=1}^{n} x_k^2 
        - 2 \bar{x} \frac{x_1 + x_2 + \ldots + x_n}{n} \cdot n + n \cdot \bar{x}^2 = \\
    &= \sum\limits_{k=1}^{n} x_k^2 - 2 \bar{x}^2 \cdot n + n \cdot \bar{x}^2 =
        \sum\limits_{k=1}^{n} x_k^2 - n \cdot \bar{x}^2, \text{ c.n.d. } \checkmark
\end{align*}

\noindent \textbf{Podpunkt (b):}
\begin{align*}
    \sum\limits_{k=1}^{n} (x_k - \bar{x})(y_k - \bar{y}) &= 
        \sum\limits_{k=1}^{n} \left(
            x_k y_k - \bar{x} y_k - \bar{y} x_k + \bar{x} \bar{y}
        \right) = \\
    &= \sum\limits_{k=1}^{n} x_k y_k - \sum\limits_{k=1}^{n} \bar{x} y_k
       - \sum\limits_{k=1}^{n} \bar{y} x_k + \sum\limits_{k=1}^{n} \bar{x} \bar{y} = \\
    &= \sum\limits_{k=1}^{n} x_k y_k - \bar{x} \sum\limits_{k=1}^{n} y_k
       - \bar{y} \sum\limits_{k=1}^{n} x_k + n \bar{x} \bar{y} = \\
    &= \sum\limits_{k=1}^{n} x_k y_k - \bar{x} n \bar{y} - \bar{y} n \bar{x} + n \bar{x} \bar{y} =
       \sum\limits_{k=1}^{n} x_k y_k - n \bar{x} \bar{y}, \text{ c.n.d. } \checkmark
\end{align*}

\newpage
\noindent \textbf{Zadanie 8} \newline
Dane są wektory $\vec{\mu}, X \in \mathbb{R}^n$ oraz macierz $\Sigma \in 
\mathbb{R}^{n \times n}$. Niech $S = (X - \vec{\mu})^T \Sigma^{-1} (X - \vec{\mu})$
oraz $Y = A \cdot X$, gdzie macierz $A$ jest odwracalna. Sprawdzić, że
\[
    S = (Y - A\vec{\mu})^T \left(A \Sigma A^T\right)^{-1} (Y - A\vec{\mu})
\]

\noindent Przy rozwiązywaniu tego zadania skorzystajmy z dwóch ważnych własności
macierzy transponowanych, jak i odwracalnych:
\begin{enumerate}
    \item $(AB)^T = B^T A^T$, podobnie $(ABC)^T = C^T B^T A^T$,
    \item $(AB)^{-1} = B^{-1} A^{-1}$, podobnie $(ABC)^{-1} = C^{-1} B^{-1} A^{-1}$.
\end{enumerate}

\noindent Działania rozpoczniemy od końca, a następnie przekształcając równanie
dojdziemy do postaci wyjściowej. Rozpiszmy więc:
\begin{align*}
    S   &= (Y - A\vec{\mu})^T \left(A \Sigma A^T\right)^{-1} (Y - A\vec{\mu}) = \\
        &= (A X - A\vec{\mu})^T \left(A \Sigma A^T\right)^{-1} (A X - A\vec{\mu}) = \\
        &= (A (X - \vec{\mu}))^T \left(A \Sigma A^T\right)^{-1} (A (X - \vec{\mu})) = \\
        &= (X - \vec{\mu})^T \underbrace{
            \underbrace{A^T (A^T)^{-1}}_{I}
            \Sigma^{-1}
            \underbrace{A^{-1} A}_{I}
        }_{\Sigma^{-1}} \ (X - \vec{\mu}) = \\
        &= (X - \vec{\mu})^T \Sigma^{-1} (X - \vec{\mu}),
            \text{ c.n.d. } \checkmark
\end{align*}


\end{document}